%   Filename    : chapter_5.tex 
\chapter{Conclusion}
%This chapter  summarizes your SP and provides conclusions regarding your results and analyses.  Provide recommendations on what ought to be done with your SP or provide further directions on the topic you covered.
This chapter presents the conclusion and recommendations of the study. The conclusion points out the important results based on the findings and validation of the people involved. Recommendations are suggestions on what could be done with the special problem covered in the study.

\section{Conclusion}

Gamification is a creative way that motivates more users to engage in a system. It's an effective instrument that motivates players to do tasks that they would not normally do in the existing systems. Giving users a fulfilling and rewarding experience will not only entertain them, but it will also give them something to anticipate and drive them to return to the system repeatedly. Towards this goal, this study has taken the initiative to apply gamification in an animal rescue system partnered with an existing rescue center in Aklan, which is the Aklan Animal Rehabilitation and Rescue Center.

To conclude, the system developed includes a gamification feature in a functioning web-based rescue system that allows a user to create an account and own a virtual pet that earns virtual points everytime the user engages in the system by donating, adopting, and volunteering at the partnered organization, the Aklan Animal Rehabilitation and Rescue Center. 

\section{Recommendation}

For the study's future development, the following are some of the recommendations:

\begin{enumerate}
	\item Some features that were not included in the final version of the system could be implemented by the developers in the future.
	
	\item The researchers recommend that the system's user interface be improved by making it more gamified-looking in order to make it more appealing and engaging to users.
	
	\item The researchers recommend developing a mobile version of the system since a web-based system always requires an internet connection to be accessed. 
	
	\item The researchers want to recommend the study's concept to the AARRC so that the rescue center's existing web-based rescue system, similar to Rescue Pets, can be gamified. Furthermore, AARRC is advised to follow the study's objectives and functions if the organization wishes to apply gamification in its existing web-based system.
	
\end{enumerate}








