%   Filename    : chapter_4.tex 
\chapter{Research Methodology}
This chapter discusses the activities and methodologies employed by the researchers in conducting the study. These include the design of the study, persons involved, technical and software specifications, testing and operation, system implementation, methods and procedures.

\section{Research Activities}

\subsection{Design of the Study}

Aklan Animal Rehabilitation and Rescue Center is the province of Aklan’s only
animal shelter. This research focuses not only on the development of an animal
system, but also on the incorporation of gamification and log-in features to it.

Merriam-Webster defines gamification as the process of incorporating games
or game-like elements into something (such as a task) in order to increase participation.[
10] Zichermann and Cunningham (2011) believe that this framework
for understanding gamification is both powerful and adaptable, as it can be easily
applied to any problem that can be solved by influencing human motivation and
behavior .[11]

The agile methodology is used by the researchers in the development of the
system. Agile Methodology is a practice that promotes continuous testing and
development throughout the development lifecycle of the system. In the Agile
model of software testing, both development and testing activities are carried out
concurrently. It is beneficial because the chosen animal rescue center will be given
frequent and early opportunities to inspect the product and make judgments and
revisions to the system before it will be implemented. [12].

\subsection{Persons Involved}

ADVISER. Being the co-author of the study, this person helps and supports
the researchers on the duration of the conduct of the study.

RESOURCE PERSON. Representative from Aklan Animal Rehabilitation and
Rescue Center is the resource person to ensure that the system is appropriate for
the users and the information is validated from the organization.

PROGRAMMER. The programmer develops the appropriate system to perform
methods and generate results in applying the gamification and log-in features.

RESEARCHERS. The researchers perform the handy activities from inquiry,
consultation, data gathering, system development, testing until the study is accomplished.

\subsection{Technical and Software Specifications}

Various software applications used for the study are considered during
the planning and development phase conducted by the researchers.

\begin{itemize}
	\item Microsoft Windows 10 for operating system - this tool was used as the main
	operating system for developing the system.
	\item TexStudio and Microsoft Word for Documentation - this tools were used for
	documentation purposes. They were used for the compilation of initial drafts and final copy of
	the SP document.
	\item Zoom application for consultations and meeting with the adviser - since
	face-to-face is not possible, this application was used by the researchers for
	consultations and meetings with the adviser.
	\item Github - Github was used for version control of the system and for easier
	collaboration. Github lets the researchers work together remotely.
	\item Adobe XD - Adobe XD is a prototyping tool for user experience and
	interaction designers. The tool was used for the initial prototype of the
	system.
\end{itemize}

\subsection{Testing and operation}

This is the phase that ensures the system’s operability and functionality.
Testing is necessary to elicit supplementary details and information to enhance
and improve the system.

People involved in the testing were asked to provide suggestions and recommendations
for enhancing and improving the system.

The study is tested by the following users: a) AARRC representative b) researcher’s
adviser c) researchers d) developers and e) random users.

\subsection{System Implementation}

The system’s possible implementation, as envisioned by the researchers, is
web-based. The mobile development of the system is further pursued based on
the researchers’ decision or can be done by future researchers who will conduct the
same study. Because the study’s focus is on the application of gamification and
log-in features to the animal rescue system, the study will be implemented after
careful and successful testing and operation of the system by domain experts.

\subsection{Methods}
\subsubsection{Federated login}

The internet is a vast world, and its population of users just grows dayby-
day. This is more prevalent as the pandemic caused lockdowns that brought
everything online. People create different user accounts, social media, educational
sites, lifestyle applications, and many other more. This results in many accounts
signing in credentials to remember, and we all know how passwords can be a
pain, we just can’t help but forget it. Some users may opt to have the same
password on multiple different accounts just so that they will not forget it, which
is a big security risk. One solution to avoid this problem without sacrificing the
safety of the user is by using federated login. Federated login means that the user
authenticates by using a third-party service such as Google or Facebook, besides
others, and not reentering any credentials or profile information.

\subsubsection{Account Creation}

\subsection{Procedures}

\subsubsection{Integration of Federated Login}

We will use Google sign-in as a federated login tool for the website. According to
Google, Google sign-in manages the OAuth 2.0 flow and token lifecycle, simplifying
the integration with Google APIs. OAuth 2.0 is the industry-standard protocol
for authorization. The steps and guide below will be taken from Google’s own
developer website, https://developers.google.com. First step is to create authorization
credentials, any application that uses OAuth 2.0 to access Google APIs
must have authorization credentials that identify that application to Google’s
OAuth Server (Google. (n.d.).

1. Go to the Credentials page, https://console.developers.google.com/apis/credentials.

2. Click Create credentials → OAuth client ID.

3. Select the Web application type.

4. Name your OAuth 2.0 client and click Create.
After configuration is complete, take note of the client ID that was
created. You will need the client ID to complete the next steps. (A client
secret is also created, but you need it only for server-side operations.)

5. Load the Google Platform Library.

6. Specify the app’s client ID created for the app in the Google Developers
Console with the google signin-client-id meta element.

7. Add a Google sign-in button, the default Google sign-in button that uses
the default setting needs to add a div element with the class g-signin2 in
the sign-in page.

8. Get profile information by using the getBasicProfile() method.
Enabling user to sign out of the web app without signing out of Google
by adding a sign-out button or link to the web site. To create a sign-out
link, attach a function that calls the GoogleAuth.signOut() method to the
link’s onclick event.

\section{Calendar of activities}

The researchers scheduled the activities on the duration of the conduct of the study, as shown in Table \ref{tab:timetableactivities}. Each bullet represents approximately one week worth of activity.

%
%  the following commands will be used for filling up the bullets in the Gantt chart
%
\newcommand{\weekone}{\textbullet}
\newcommand{\weektwo}{\textbullet \textbullet}
\newcommand{\weekthree}{\textbullet \textbullet \textbullet}
\newcommand{\weekfour}{\textbullet \textbullet \textbullet \textbullet}

%
%  alternative to bullet is a star 
%
\begin{comment}
	\newcommand{\weekone}{$\star$}
	\newcommand{\weektwo}{$\star \star$}
	\newcommand{\weekthree}{$\star \star \star$}
	\newcommand{\weekfour}{$\star \star \star \star$ }
\end{comment}



\begin{table}[ht]   %t means place on top, replace with b if you want to place at the bottom
	\centering
	\caption{Timetable of Activities} \vspace{0.25em}
	\begin{tabular}{|p{1.8in}|c|c|c|c|c|c|c|c|} \hline
		\centering Activities (2021-2022) & Sep   & Oct & Nov & Dec & Jan & Feb & Mar & Apr \\ \hline
		Identify possible SP topics      &  \weekone  &  &  &  &  &  &  & \\ \hline
		Proposal of chosen SP topic to the adviser &  & \weekone &  &  &  &  &  & \\ \hline
		Approval from the adviser of the chosen SP topic       &  & \weekone &  &  &  &  &  & \\ \hline
		Presentation of use cases and workflow of the system    &   & \weekone &  &  &  &  &  & \\ \hline
		Research for related literatures  &   & \weekone & \weekfour &  &  &  &  & \\ \hline
		Making of research proposal (First draft) & \weekone  & \weekfour & \weekone  &  &  &  &  & \\ \hline
		Development of the System &  & \weekone & \weekfour & \weektwo & \weektwo & \weekfour &  & \\ \hline
		Testing and operation &  &  &  & \weektwo &  &  & \weekfour & \\ \hline
		Analysis and Interpretation of results &   &  &  &  &  & \weekfour & \weekfour & \\ \hline
		Discussion of results &  &  &  & \weekfour &  & & \weekfour & \weekfour \\ \hline
		Documentation & \weekone  & \weekfour & \weekfour & \weekfour & \weekfour & \weekfour & \weekfour & \weekfour \\ \hline
	\end{tabular}
	\label{tab:timetableactivities}
\end{table}


