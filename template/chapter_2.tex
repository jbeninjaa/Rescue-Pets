%   Filename    : chapter_2.tex 
\chapter{Review of Related Literature}
\label{sec:relatedlit}


\begin{comment}
%
% IPR acknowledgement: the contents withis this comment are from Ethel Ong's slides on RRL.
%
Guide on Writing your RRL chapter
 
1. Identify the keywords with respect to your research
      One keyword = One document section
                Examples: 2.1 Story Generation Systems
			 2.2 Knowledge Representation

2.  Find references using these keywords

3.  For each of the references that you find,
        Check: Is it relevant to your research?
        Use their references to find more relevant works.

4. Identify a set of criteria for comparison.
       It will serve as a guide to help you focus on what to look for

5. Write a summary focusing on -
       What: A short description of the work
       How: A summary of the approach it utilized
       Findings: If applicable, provide the results
        Why: Relevance to your work

6. At the end of each section,  show a Table of Comparison of the related works 
   and your proposed project/system

\end{comment}

\section{The Rise of Animal Welfarism}
~~~~The 1944 legal reforms were portrayed as a set of rules in the government proposal response and resolution to a long list of animal-treatment-related issues. The old anti-cruelty legislation law from 1921 should be replaced with an animal protection law that provides a comprehensive regulatory framework to set standards for preventing animal mistreatment in a broader sense.
The primary social function of the old law was to regulate excessive animal use and to police the borders of animal exploitation. The use of animals had changed and evolved because of the anti-cruelty regime. Things like the expansion of large-scale breeding and farming, longer and more frequent animal transports, an increase in the number of animal experiments, and so on were bound to raise new questions about the systemic nature of animal abuse and was the real problem that needed to be addressed (Svard, 2015).

Svard(2015) believes that the 1944 reform's most important contribution was not simply expanding legal protection for nonhuman animals, but also bringing closure and political control to a progressively broken discursive regime. The 1944 reforms achieved what proponents of the current system had fought for in the past: a fully contextualized problematization that would have been troubled by norms contradictory to the human-animal relationship[2].

The 1944 reform advocates to broaden the legal protection to animals by legalizing the punishment to animal abusers. However, the present study focuses on rescuing and taking care of the stray animals found in the streets by building a web-based system responsible for these activities.


\section{Philippine Animal Welfare Society}

~~~~The organization is dedicated to animal protection and humane treatment. The organization advocates for the elimination of animal cruelty and pet neglect. In addition, the organization works to educate and disseminate information about animal welfare and protection.

People can donate cash or in-kind to the organization, they can adopt a pet, and the organization rescues animals and rehabilitates them. The organization does not accept previously owned pets.
Humane education, where educational tours, interviews, school visits, and seminars are provided for free, spay and neuter surgeries, where a low-cost clinic is available to the public, animal therapy, and disaster relief operations are among the other programs offered by the organization[3].

The organization uses a web-based system to post and record all activities. The system includes options for adopting, donating, and volunteering. The current study will also create a system for a local organization concerned with animal welfare and protection. The Aklan Animal Rehabilitation and Rescue Center has been chosen as the local organization.

\section{The Albert Foundation}

~~~~The Albert Foundation/Stiching Albert is a registered charity in the Netherlands that collects donations and sponsors the Aklan Animal Rehabilitation and Rescue Center. It was founded by Michel and Neressa Van Der Kleij. The foundation has a web-based system in place for AARRC to post information about the rescue center. The AARRC rescues animals for them to be sponsored or adopted and live a better life. Most of the animals came from Aklan's streets, but only a few came from people who no longer wanted to care for their animals because dumping animals is strongly discouraged by the center.

Supporting the local community by purchasing and hiring locally is an integral feature of the AARRC philosophy of "Animal Welfare Equals Human Welfare." We cannot exist without our fellow beings, and they require our assistance. We start serving the human community by improving public health and safety through our work with stray animals[4].

The web-based system is simply one in which you do not need to log in because you can access everything about the AARRC without creating an account. The system contains information about the founder, the mission and vision, the programs that have been implemented, and instructions on how to donate. Their email, Facebook, Twitter, Instagram, and YouTube accounts are also linked by the system. The system is available in English, Dutch, and Spanish.

The current study aims to adapt AARRC's current system while adding new features such as account creation and having a virtual pet. The current study aims to add account creation and gamification features to the system to increase user engagement. By including account creation, users will be able to create an account before engaging in any transactions with AARRC. Account creation also aids in the monitoring of users who have a virtual pet that is automatically gifted to them after creating an account.

\section{Gcash Forest}

~~~~Gcash forest is a GCash gamified in-app feature that allows users to plant virtual trees via the Gcash app. Green Heroes are people who use the Gcash forest. After three days, the users collect Green Energy, which is then used to plant trees. Previously, the period for planting trees was only 24 hours.

The feature aims to help save the environment through the Gcash app, which has 20 million users. Despite the situation, Martha Sazon, CEO and President of Gcash, believes that the changes will allow users to plant more trees because digital transactions are becoming the new norm[5].

According to Anthony Tomas, Mynt CEO, Gcash forest addresses the issue of citizens who want to help save the environment but don't know-how. Making a difference is much easier now that they can plant trees with their smartphones. The feature is modeled after Alipay Ant Forest, which is run by Ant Financial. Gcash Forest worked in partnership with the Department of Environment and Natural Resources (DENR), the World Wildlife Fund (WWF), and the Biodiversity Finance Initiative (BIOFIN). The DENR made available land resources, the WWF equipped trees and workforce, and BIOFIN provided monitoring expertise [6].

The present study will use a gamified feature as well, but instead of virtual trees, a virtual pet will be used because the system is for animal rescue. The Gcash forest harvests green energy, whereas the present study employs digital points to fund the virtual pet. The virtual pet's goal is to validate every user's help and support in saving and protecting animal strays via the system that will be built.

\section{Gamification}

~~~~Merriam-Webster defines gamification as the process of incorporating games or game-like elements into something (such as a task) in order to increase participation.[7]  A study titled "Does Gamification Work?" was conducted. — A Literature Review of Empirical Studies on Gamification" (Hamari et al., 2014) examined 24 scientific studies that shared a high-level research question (not always clearly stated): Does gamification work? The various studies examined use a variety of gamification mechanics . According to the findings, the majority of the studies examined yielded a positive outcome and benefits. It is also important to note that some of the studies had limitations, such as a limited sample and a lack of control groups.[7] 

Gamification is the method of using gameplay mechanics and game thinking to user engagement and resolve issues (Cunningham et al., 2011). It is a powerful tool that encourages players to complete a task or use customer support that they would not have done normally. Giving users a satisfying and rewarding experience will not only provide them with entertainment, but will also give people something to look forward to and encourage them to use the service repeatedly.

Moreover, Zichermann and Cunningham think that the player is the root of gamification. In a gamified system, the user’s or the player’s motivation to use the system drives the outcome. Therefore, understanding player motivation is paramount to building a successfully gamified system (Cunningham et al., 2011). A good video game values its player more than others, a developer listens to the opinion of the player of the game and provides patches for the game to be better. This should also be adapted in any gamified system as it is a powerful tool to gain more engagement and for the long-term success of the system. Predicting how the “players” of the system would act will allow the builder to reach the main objectives of the system.[8]

The current study will employ gamification features to improve user engagement. This will be accomplished by having a virtual pet after logging in. Each transaction made by the users equates to a certain number of reward points earned by the virtual pet. Once the virtual pet has accumulated a certain number of points, it will level up and change appearance. 

\section{Automatic Login of Account}

~~~~The internet is a vast world, and its user base is growing by the day. This is more common now that the pandemic has resulted in lockdowns that have brought everything online. People create various user accounts, social media sites, educational sites, lifestyle applications, and a variety of other things. As a result, there are many accounts signing in credentials to remember, and we all know how passwords can be a pain, and we can't help but forget them. Some users may choose to use the same password on multiple accounts in order to avoid forgetting it, which poses a significant security risk. Using federated login is one way to avoid this problem without jeopardizing the user's safety. Federated login means that the user authenticates by using a third-party service, such as Google or Facebook, rather than entering credentials or profile information again. [9]

A lot of web-based and even mobile applications make use of account signing. That's why we often forget our credentials. The present study will make use of the automatic log-in of accounts to enable the users to log-in their account easily without the need to input credentials. 











